\documentclass[a4paper,12pt]{article}
\usepackage[style=apa]{biblatex}
\addbibresource{./SS2040_A01_public_opinion_and_mass_media.bib}
\usepackage{ragged2e}
\usepackage{graphicx}
\usepackage[
    colorlinks
]{hyperref}

\hypersetup{
    pdfinfo = {
        Title = Public Opinion and Mass Media,
        Author = {Saad Nawaz Ghauri},
        Subject = Mass Communication assignment,
        Keywords = {mass media, public opinion, propaganda},
        Producer = LuaLaTeX
    },
    bookmarksopen = true,
    pdfmenubar = true,
    pdfnewwindow = true,
    pdftoolbar = true,
    verbose = true,
}

\author{Saad Nawaz Ghauri}
\title{Public Opinion and Mass Media}

\begin{document}
\maketitle

\begin{Center}
\includegraphics{nuces_logo.png}
\end{Center}

\begin{Center}
    Section: \emph{BCS-4A} \\
    Role Number: \emph{23L-1007} \\
    Course Code: \emph{SS-2040} \\
    Course Title: \emph{Mass Communication} \\
\end{Center}

\begin{Center}
    School of Computing \\
    National University of Computer and Emerging Sciences, \\
    Lahore Campus \\
\end{Center}

\newpage

\tableofcontents
\newpage


\section{Introduction}
This article will explore the complex relationship between mass media and public opinion, how they influence each other.

\section{Definition}
Before attempting to understand or measure the way public opinion and mass media interact, it would be useful to define what these terms are.
\subsection{Public Opinion}
Public opinion is the aggregate of the views, opinions, attitudes, and beliefs held by the majority of the public,
when it comes to a topic, or at least by a significant part of society~\parencite{davison2024public}.
\subsection{Mass Media}
Mass media is any kind of media that is made, or meant, to reach a large audience~\parencite{duignan2024mass}.
Just like the term media itself, mass media can include numerous types of media.
Mass media can be classified in to three categories, traditional media and new media~\parencite{manohar2008different}.
\subsubsection{Print Media}
Print media includes all kinds of printed material, aimed at large audiences, such as newspapers, magazines, and newsletters.
It is not limited to news media, but also advertisement material, leaflets, flyers, and pamphlets.
\subsubsection{Electronic Media}
\subsubsection{New-age Media}

\section{Main Theme}
\subsection{Positive Effects}
\subsection{Negative Effects}
\subsection{Examples}

\section{Conclusion}

\newpage
\printbibliography

\end{document}
